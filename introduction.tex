\vspace{0.1in}
\section{Introduction}
\label{sec:introduction}

Ranking physical establishments, such as restaurants and bars, is a problem
of significant commerical importance.
An understanding of an establishment's quality has wide applicability in improving
local search, recommendation systems, and mapping software.
However, broadly quantifying establishment quality at scale is a particularly difficult
research problem at the heart of geographical information retrieval, machine learning, and
natural language processing.

Questions such as: "Where is the best pizza in New York City?" can only be answered by humans.
Humans evaluate establishments in ways that machines could never achieve.
They can taste the foods, experience the environment, and reflect on the service.
If a mapping service were equipped with this type of information for all restaurants,
it would be able to offer much higher quality results when users search for "pizza" or
other similarly generic terms.

Obtaining quality judgements for all physical establishments is a challenging problem,
First of all, there are many complicated facets to determining the quality of a place.
For example, one may judge an establishment on quality of service, decor,
ambiance, hygiene, and food.
Tackling the problem at scale is also particularly difficult.
In addition to the many establishments that exist, new establishments are
being opened around the world constantly.
In order to leverage this type of information for local search results,
an enormous amount of data must be collected on all relevant establishments in order to 
achieve an acceptable level of confidence in the ranking.
Lastly, these various data sources must be distilled in a way to be able to compare
two establishments.
For example, if one restaurant as 100 check-ins and a 3-star rating and
another restaurant has 10 check-ins and a 4-star rating, which is better?

Traditionally, this task has been accomplished by crowdsourcing.
There exist a variety of websites (e.g. Google Maps, Yelp, TripAdvisor)
that allow users to review and rate their previous experiences at an establishment.
Recommendation engines then may rely on a particular signal such as total check-in count
to use for recommending places for you to go.
In some cases, these services may use a simple heuristic amongst various data sources
such as average rating and review count.
Unfortunately, any individual signal tends to be a very noisy and a generally sparse data set,
over the set of all establishments.
Crowdsourced data also tends to be largely a function of current consumer behaviors and technology.
Location check-in services experienced incredible growth in the recent few years.
However, this growth was disproportionate depending on the location
and type of establishments.
Thus, extracting what these signals mean in terms of quality and popularity can a highly
complex problem, with many interdependencies between signals.

While new forms of social media data presents challenges for recommendation engines,
they also exhibit exciting opportunities for gaining rich insights into consumer behavior.
This data includes user-generated content (e.g. reviews, check-ins),
as well as automatically generated content (e.g. check-in time).
Users are expressing their satisfaction and dissatisfaction of places in
ever-growing ways on the Internet.
While any individual signal may be noisy and unreliable, the collective group
of all different types of social media provides revealing information
into consumer opinions.

In this paper, we exploit this rich space of features with \emph{SolocoRank},
a query-independent prediction of quality.
\emph{SolocoRank} is a machine learning model trained on a number of Web-based signals,
including a number of new forms of social media.
In addition to traditional signals such as the establishment website's PageRank \cite{pagerank},
we also incorporate features derived from reviews, location check-in data,
mentions on micro-blog posts, and photo/video counts of a place.
This model is then used to classify all establishments in the United States along a 0-30 scale.
The \emph{SolocoRank} of a place does not depend on the intended search query,
but could be used to help determine the ultimate ordering of search results.
Many different signals may be used to determine the relevance 
of each search result \cite{bartell1994}, one of which may be \emph{SolocoRank}.

The contributions of this paper as follows:
\squishlist
  \item We pose the problem of producing a score of quality
  as a machine learning task, where each place has features
  derived from various crowdsourced social media data sources 
  (Section \ref{sec:setup}).
  \item We propose a general classification framework,
  suitable for learning establishment quality scores
  from social media (Section \ref{sec:design}).
  \item We evaluate our proposed classification framework
  on real data from a variety of real-world web services
  (Section \ref{sec:evaluation}).
\squishend
We conclude with a discussion of the implications of our findings and 
directions for future work in Sections \ref{sec:discussion} and \ref{sec:conclusion}.
