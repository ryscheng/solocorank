\begin{abstract}
User-contributed Web data contains rich information about
the physical establishments in the real-world, such as restaurants and hotels.
Our goal is to leverage this growing source of data on the Web
in order to improve the quality of restaurant recommendations from local search engines.
We do so by generating a query-independent score of quality for all establishments in the US.
Search engines that rely purely on raw user-generated reviews suffer from three major limitations.
First, the same top places are reinforced at the head of results, causing bias towards places
that were introduced first. 
Second, user-generated review scores offer very little insight into the establishment's quality.
Scores are limited to fairly coarse resolution, such as discrete values from 0 to 5,
and research has shown that scores tend to be biased towards high scores.
Lastly, user-generated reviews are notoriously noisy, as each person may hold very different
standards in what the value of each star means.
User-generated reviews alone act as a weak signal for indicating the quality of an establishment,
and search engines that use them naively can lead to poor results.
Instead, we demonstrate that many of these problems can be mitigated
by leveraging editorial review services, such as Zagat.

We built a new learn-to-rank system, named \emph{SolocoRank}, trained on
scores from editorial review services to more accurately rank all establishments.
\emph{SolocoRank} leverages a variety of signals on the Web, including 
new social media data sources such as check-ins and microblogs.
Our approach has shown to accurately predict editorial review scores.
When trained on Zagat scores, we show that \emph{SolocoRank} exhibits significant performance
gains in ranking establishments, even for the long-tail
of lower quality establishments.

\end{abstract}
