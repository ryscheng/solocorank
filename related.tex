 \section{Related Work}
\label{sec:related}
The rise of social media on the Internet and its impact has been extensively studied.
Social media provides unique insights into real-world behaviors and opinions.
Studies have shown that users are increasingly relying on social media
in vacation planning and social media takes up a growing part of Web search results \cite{xiang2010}.
Likewise, websites that can leverage social media to recommend restaurants
have been shown to hold real economic power.
For example, it has been shown that a one-star increase in Yelp ratings
leads to a 5-9\% increase in revenue in the state of Washington \cite{luca2011reviews}.

There has been extensive study into extracting information from user-generated reviews.
Natural language processing techniques, such as sentiment analysis, 
can be used to extract favorability from raw text \cite{dave2003,nasukawa2003}.
These techniques have been applied to reviews to produce stronger models 
of consumer opinion \cite{carrillo2011,shamshurin,morinaga2002}.
Sentiment analysis models have grown to become very sophisticated, using machine learning
techniques to model cross-sentence context \cite{Pang:2004:SES:1218955.1218990}.

Ganu et al found that by using sentiment analysis on raw review text,
they can generate higher quality scores for use in recommendation systems,
when compared to systems that just used star ratings \cite{ganu2009}.
\emph{SolocoRank} does not incorporate any signals generated from raw review text.
Instead, we use a variety of signals derived from aggregated counts of rating stars
as described in Section \ref{sec:features}.
In the future, sentiment analysis techniques can be used to strengthen the
\emph{SolocoRank} model, improving on our results. 

Lala et al also investigated the use of social media to improve
restaurant recommendations \cite{lalamine}.
Their system collects an individual user's rating of select restaurants.
Then, item-based collaborative filtering is used on review text 
to find personalized recommendations for that particular user \cite{sarwar2001}.
\emph{SolocoRank} aims to provide a more general framework for predicting 
review scores, incorporating new forms of social media such as check-ins.

Kawamae also used collaborative filtering techniques on reviews,
but in order to predict future reviews from a particular review author \cite{kawamae2011}.
He used latent evaluation topic models to differentiate an author's preference.
These models tracked the variety of words that distinguish the author's
attitude, which can then be used to find other like-minded reviewers.

Social media has also been used to generate quality scores in search engines.
SocialPageRank \cite{bao2007} calculates the popularity of webpages using
social annotations from websites like Del.icio.us.
They found that high quality webpages are generally bookmarked by
up-to-date users, using hot annotations.
Their algorithm calculcates a quality signal from these social annotations,
boosting the performance of their search engine.

\emph{SolocoRank} is largely inspired by various systems that have been proposed to use
social media and other Web signals to improve the ranking quality of a search engine \cite{ganu2009,bao2007,xiang2010}.
However, there are significant differences.
Studies have shown that on their own, these Web signals offer only limited understanding into
the quality of an establishment.
For example, studies have shown that click-through data is not reliable for obtaining
absolute relevance judgements \cite{jain:imagereranking}.
Rather than using any particular Web signal directly, \emph{SolocoRank} trains a model that aims to
accurately predict the editorial rating of an establishment, using a variety of different
data sources at our disposal.

Machine learning has become a popular mechanism for generating
quality scores for search engine ranking \cite{boyan,burges2005}.
Richardson et al used machine learning to rank
webpages that use features beyond the link-structure
of the Web \cite{richardson2006,amento2000}.
This work was shown to outperform PageRank \cite{pagerank} in 
Web search quality.

Query-independent signals have been studied as effective mechanisms to 
improve ranking in the context of webpages \cite{upstill2003}.
Craswell et al introduced mechanisms to transform query-independent
signals into effective features for learn-to-rank systems \cite{craswell2005}.
Dalvi et al introduced adversarial classification, which allows
a system to achieve robust performance, even when adversaries try to game
the system \cite{dalvi2004}.
\emph{SolocoRank} draws much inspiration from the existing learn-to-rank literature,
and applies them to a new space~\cite{boyan,burges2005,richardson2006,amento2000,craswell2005,upstill2003,cao2007learning,radlinski2005query,liu2007letor,xia2008listwise,duan2010empirical,agarwal2006learning,kulis2006learning}.

Geographic search engines use a variety of mechanisms to recommend places.
Kato et al proposed an alternative user interface.
By selecting places in the user's hometown that they like,
the system uses various distance metrics to find similar places at the
user's current location \cite{kato:geosearch}.


